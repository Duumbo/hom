\Boxed{L'effet Hong-Ou-Mendel}

\question{A}{
    Introduction
}

Une des plus grandes avancées de la thermodynamique est la démonstration de ce
que l'on nomme le théorème spin-statistique. Ce théorème a pour effet de
séparer les particules en deux catégories, soit les bosons ou les fermions.\\

Les propriétés de chaque type sont assez différentes, tel que les personnes
derrière les noms de celles-ci. On disait d'Enrico Fermi qu'il était un peu
une tête brûlée; il est le père de la fission nucléaire. Ses expériences étaient
très importantes, mais non sans risque. Une blague qui circulait dans son laboratoire
lors de son travail sur les "piles": "Si les gens pouvaient voir ce que nous
faisons avec 1.5 M\$, ils nous trouveraient fous. S'ils savaient pourquoi nous
le faisons, ils auraient raison."\cite{fermi1}\cite{fermi2}\\

D'un autre côté, Satyendra Nath Bose aurait été qualifié autrement. Après avoir
redérivée les coefficients de la loi de Planck et envoyé ses résultats à Einstein,
Bose est allé en Europe pour en apprendre plus sur la radioactivité dans le
laboratoire de Curie. Le laboratoire étant français, Curie lui aurait fait un
long exposé sur l'importance de savoir parler français. Il aurait ensuite appris
le français, mais n'en aurait jamais parlé à Curie, qui l'aurait sans doute accepté
comme assistant.\cite{bose1}\cite{bose2}.

\question{B}{
    Partie 1: Symétrisation et Anti-Symétrisation
}

Au même titre que les deux personnes d'après lesquelles ces particules tiennent
leur nom, les bosons et les fermions ont des propriétés opposées.

Prenons pour exemple le problème suivant, deux particules identiques entrent en collision
puis partent de leur côté. Comment savoir quelle trajectoire la particule à prise?
On ne peut pas le savoir, c'est ce que l'on nomme la dégénérescence d'échange\cite{senech}.
Il y a moyen de passer à côté de ce problème. En mécanique quantique, un outil
qui permet de donner un sens à la position d'une particule est la fonction
d'onde. On relie la probabilité que la particule se trouve dans un intervalle
de position comme

\begin{align}
    P(I)&=\int_I\abs{\psi(x)}^2\dd x\qquad I,\ \text{un intervalle}
\end{align}

On peut représenter mathématiquement la dégénérescence d'échange comme le fait
que la probabilité avant l'échange doit être la même que la probabilité après
l'échange. Il apparaît donc deux solutions possibles, soit que la fonction
d'onde ne change pas sous l'échange, soit que la fonction d'onde prenne un signe
négatif. C'est ici que les fermions, ceux que la fonction d'onde prend un signe
négatif, se démarquent des bosons, ceux que la fonction d'onde reste inchangée.

\question{C}{
    Partie 2: Effet Hong-Ou-Mendel
}

Une petite description de l'effet HOM. Comment faire l'expérience qui permet
de mesurer l'effet HOM.
Présentation d'un montage expérimental\cite{hom1}\cite{hom2}\cite{hom3}.
L'élément clef dans l'effet HOM, c'est le fait que la lame partiellement réfléchissante
l'est à 50\%.

Pour des bosons, la transformation se fait

\begin{align}
    a_1^\dagger&=\frac1{\sqrt{2}}\qty(b_1^\dagger+b_2^\dagger)\\
    a_2^\dagger&=\frac1{\sqrt{2}}\qty(b_1^\dagger-b_2^\dagger)\\
    a_1^\dagger a_2^\dagger&=\frac12\qty(\qty(b_1^\dagger)^2-\qty(b_2^\dagger)^2+b_1^\dagger b_2^\dagger-b_2^\dagger b_1^\dagger)\\
    [b_1^\dagger,b_2^\dagger]=0\\
    a_1^\dagger a_2^\dagger&=\frac12\qty(\qty(b_1^\dagger)^2-\qty(b_2^\dagger)^2)
\end{align}

Pour des fermions, la transformation se fait

\begin{align}
    c_1^\dagger&=\frac1{\sqrt{2}}\qty(d_1^\dagger+d_2^\dagger)\\
    c_2^\dagger&=\frac1{\sqrt{2}}\qty(d_1^\dagger-d_2^\dagger)\\
    c_1^\dagger c_2^\dagger&=\frac12\qty(\qty(d_1^\dagger)^2-\qty(d_2^\dagger)^2+d_1^\dagger d_2^\dagger-d_2^\dagger d_1^\dagger)\\
    \qty(d_i\dagger)^2&=0\\
    \{d_1^\dagger,d_2^\dagger\}=0\\
    c_1^\dagger c_2^\dagger&=d_1^\dagger d_2^\dagger
\end{align}

\question{E}{
    Conclusion
}

Pour conclure, l'apparition de l'effet HOM est directement liée à l'existence
des bosons. On peut utiliser l'effet HOM pour quantifier la qualité d'une source
de photons unique. Est-ce que l'effet HOM existe pour les électrons?


\clearpage

